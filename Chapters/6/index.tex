\chapter{Závěr}
\label{sec:Chapter6}
V rámci této práce byl prozkoumán postup detekce modelu Hot Rod 42022, následná lokalizace jeho klíčových bodů pomocí sítě YOLOv11-\textit{pose} a odhad pózy prostřednictvím algoritmů PnP za využití metody RANSAC. Výsledkem je implementace funkčního přístupu pro odhad pózy, fungujícího jak na jednotlivých snímcích, tak i ve videosekvencích, a to s úspěšnými výsledky ve většině případů. Funkčnost řešení byla ověřena v grafickém rozhraní, které slouží k vizualizaci relativní pozice kamery vůči objektu zájmu.

Použitý model YOLOv11-s dosáhl metriky mAP@0,5 OKS ve výši 0,73 a mAP@0,5:0,95 ve výši 0,435, přičemž hodnoty metrik naznačují fakt, že delším tréninkem lze tohoto výsledku dále zlepšit. V rámci navazující práce lze navrhnout rozšíření rozsahu škálování zájmového modelu při generování syntetického datasetu, čímž by mohlo dojít ke zvýšení škálové invariance.