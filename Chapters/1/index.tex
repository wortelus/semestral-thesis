\chapter{Úvod}
\label{sec:Introduction}
Určování polohy objektů ve scéně patří mezi klíčové problémy současné počítačové grafiky a zpracování obrazu. Jedním z častých přístupů k jeho řešení je lokalizace klíčových bodů, tedy specifických bodů nacházejících se na objektu zájmu, které společně nesou zásadní informace o jeho tvaru, orientaci a poloze. V úspěšném případě lze odhadnout 6 stupňů volnosti (DoF) zájmového objektu.

V rámci této semestrální práce bude proveden pokus o nalezení klíčových bodů pomocí hlubokých neuronových sítí, případně detektorů, jako je YOLO – jeden z nejpoužívanějších modelů pro detekci objektů, tedy úlohu spočívající v nalezení objektů ve scéně a jejich lokalizaci pomocí tzv. ohraničujících boxů (ang. bounding boxes). Následné řešení bude evaluováno a vyobrazeno pomocí grafického rozhraní pro vizualizaci odhadu vzájemné pózy objektu ke kameře. Práce navazuje na postupy a řešení představené v mé bakalářské práci \cite{mojebp}, přičemž jejím cílem je dále vylepšit navržený přístup, ověřit jeho funkčnost na reálných snímcích a následně provést odhad polohy objektu pomocí řešení úlohy perspektivního problému n bodů.