\section{Výkon modelu}
\label{sec:Chapter51}
V rámci evaluace lokalizace byl parametr sigma pro metriku OKS nastaven na $\sigma=0.05$. Volba hodnoty je v rozsahu hodnot $\sigma$ z dat pro odhad lidské pózy COCO \cite{dutta2023oks}. Výsledky na základě testovací množiny jsou následující:

\begin{table}[hb]
    \centering
    \begin{tabular}{@{}l S[table-format=1.3] @{\hspace{2cm}} l S[table-format=1.3]@{}}
        \toprule
        \multicolumn{2}{l}{\textbf{Detekce (Box)}} & \multicolumn{2}{l}{\textbf{Odhad pózy (Pose)}} \\
        \cmidrule(lr){1-2} \cmidrule(lr){3-4}
        Metrika & {Hodnota} & Metrika & {Hodnota} \\
        \midrule
        Přesnost (P) & 0,995 & Přesnost (P) & 0,810 \\
        Úplnost (R) & 0,991 & Úplnost (R) & 0,780 \\
        mAP@0,5 & 0,995 & mAP@0,5 OKS & 0,782 \\
        mAP@0,5:0,95 & 0,909 & mAP@0,5:0,95 OKS & 0,528 \\
        \bottomrule
    \end{tabular}
    \caption{Přesnost detekce a lokalizace modelu YOLO}
    \label{tab:vysledky_test}
\end{table}

Z tabulky \ref{tab:vysledky_test} je patrné, že detekce a odhad ohraničujících boxů dosahuje vysokých hodnot všech měřených metrik přesnosti, úplnosti a mAP. Avšak výsledky jsou nejzajímavější pro odhad pózy (lokalizaci), kde průměrná přesnost nad prahem $0,5$ dosahuje zhruba $0,78$. Přísnější metrika mAP@0,5 však dosahuje zhruba přesnosti $0,53$. Po vizuální inspekci inferenčních výsledků s nízkou hodnotou mAP se dá usoudit, že na vině jsou primárně snímky s neúplným modelem v záběru. Na lokálním systému jsme dosáhli následujících průměrných inferenčních rychlostí:

\begin{table}[ht]
    \centering
    \begin{tabular}{@{}l l l r}    
    \toprule
    Předzpracování & Inference & Post-zpracování & Velikost vstupu \\
    1,0 ms & 11.8 ms & 1.8 ms & $640\times384\times3$ \\
    \end{tabular}
    \caption{Rychlost inference modelu YOLO}
    \label{tab:rychlost_yolo}
\end{table}